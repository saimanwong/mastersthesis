\begin{table}[ht!]
    \scriptsize
    \caption{Summery of networking tools from~\cite{mahalakshmi2016study}}
    \label{trexsummary}
    \begin{adjustbox}{width=1.25\textwidth, center}
        \renewcommand*\arraystretch{1.5}\begin{tabular}{| L{3cm} | L{3cm} | L{3cm} | L{3cm} | L{3cm} |}
            \hline
            \textbf{Characteristics} & \textbf{Pktgen} & \textbf{TRex} & \textbf{Ostinato} & \textbf{MoonGen} \\ \hline
            \textbf{1) Availability} &
            Open Source BSD license &
            Cisco Tool... Open Source but limited. &
            Open Source &
            Free software under the MIT license
            \\ \hline
            \textbf{2) Usage with respect to architecture} &
            Uses DPDK architecture -- Requires EAL &
            Requires extra hardware for full functionality &
            Uses DPDK architecture \newline Requires EAL &
            Uses DPDK architecture -- Requires EAL \newline Also, utilizes hardware features of Intel NICs
            \\ \hline
            \textbf{3) Cost of architecture} &
            Pktgen is free and no additional hardware required &
            Cisco UCS C220 M3 \newline Rack Server-starts at \$2,397 &
            Free with no additional hardware requirements &
            The cost of commodity hardware
            \\ \hline
            \textbf{4) Speed in terms of saturation} &
            Usually 10000 Mbps &
            Tx= up to 200Gb/sec Rx=up to 200Gb/sec &
            1/10G line rate &
            178.5 Mbps, line rate at 120 Gbit/s.
            \\ \hline
            \textbf{6) Advantage and disadvantages} &
            \textbf{Advantage:} Easily Integrable \newline
            \textbf{Disadvantage:} \newline
            \textbf{1.} All packet templates need to be available in memory \newline
            \textbf{2.} Compat problem (Has been resolved now) &
            \textbf{Advantage:} \newline
            \textbf{1.} Ready to use. \newline
            \textbf{2.} With hardware, pcap files can be altered for experiments \newline
            \textbf{Disadvantage:} \newline
            \textbf{1.} Extra hardware \newline
            \textbf{2.} Limited Functionality &
            \textbf{Advantages:} \newline
            \textbf{1.} Ready GUI. \newline
            \textbf{2.} Cross- Platform - Windows, Linux, BS and Mac  \newline
            \textbf{Disadvantages:} \newline
            \textbf{1.} Supports DPDK w limited functionality \newline
            \textbf{2.} A Work in progres with Ds &
            \textbf{Advantages:} \newline
            \textbf{1.} Supports precise and accurate timestamping and rate control \newline
            \textbf{2.} Implemented in software and runs on commodity hardware
            \textbf{Disadvantages:} \newline
            \textbf{1.} Unchecked memory accesses can lead to memory corruption
            \\ \hline
            \textbf{7) Is any additional hardware required?} &
            No &
            Yes. Cisco UCS hardware &
            No &
            Yes, commodity hardware.
            \\ \hline
            \textbf{8) Is DPDK compatible NIC required?} &
            Yes &
            Yes &
            Yes &
            Yes
            \\ \hline
            \textbf{9) Software and Hardware Requirements } &
            Ubuntu 13.10 x86\_64, kernel version 3.5.0-25, on a Westmere Dual socket \newline~\newline board running at 2.4GH, with 12GB of ram 6GB pe socket  &
            Ubuntu 14.10, X86/Intel NIC 1350,82599,XL710  &
            Windows, Mac OS X 10.5 or later, Debian 8, Ubuntu 13.10 \newline~\newline DPDK compatible NIC  &
            Windows, Mac OS X 10.5 or later, Debian 8, Ubuntu 13.10 \newline~\newline DPDK compatible NIC
            \\ \hline
            \textbf{10) L3 Forward Support?} &
            Yes &
            Yes &
            Yes &
            Yes
            \\ \hline
            \textbf{11) Scalability} &
            Scalable &
            Scalable &
            Scalable &
            Scalable
            \\ \hline
            \textbf{12) Support for L4- L7 layers} &
            Yes &
            Yes &
            Upto L5 &
            Yes
            \\ \hline
        \end{tabular}
    \end{adjustbox}
\end{table}
