\begin{table}[ht!]
    \scriptsize
    \caption{Summary of Traffic Generators Types \cite{botta2010you, molnar2013validate}}
    \label{types}
    \begin{adjustbox}{center}
        \renewcommand*\arraystretch{1.5}\begin{tabular}{| L{5cm} | L{8cm} |}
            \hline
            \textbf{Replay Engines} & Replay network traffic back to specified \gls{nic} from a file which contains prerecorded traffic, usually a pcap-file.
            \\ \hline
            \textbf{(*) Maximum Throughput Generators} & Generate maximum of network traffic with the purpose to test overall network performance, for example, over a link.
            \\ \hline
            \textbf{Model-Based Generators} & Generate network traffic based on stochastic models.
            \\ \hline
            \textbf{High-Level and Auto-Configurable Generators} & Generate traffic from realistic network models and change the parameters accordingly.
            \\ \hline
            \textbf{Special Scenario Generators} & Generate network traffic with a specific characteristic, for example, video streaming traffic.
            \\ \hline \hline
            \textbf{Application-level Traffic generators} & Generate network traffic of network applications, for example, the traffic behavior between servers and clients.
            \\ \hline
            \textbf{Flow-Level Traffic generators} & Generate packets in a particular order that resembles a particular characteristic from source to destination, for example, Internet traffic.
            \\ \hline
            \textbf{(*) Packet-Level Traffic Generators} & Generate and craft packets, usually, from layer 2 and up to 7.
            \\ \hline
        \end{tabular}
    \end{adjustbox}
\end{table}
