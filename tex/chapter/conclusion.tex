\chapter{Conclusion}\label{conclusion}

Our study evaluates the performance of the network tools Iperf, Mausezahn, and Ostinato.
We use the metrics throughput and varying packet sizes to measure the performance of these in closed-loop environments with both physical and virtual hardware, as summarized in \cref{tab:comparesummary}.
The tools operate in the user space and use the host \acrshort{os}' default network stack to craft and send traffic.
Given the tools generality, the user can efficiently conduct smaller tests and deploy them on many various systems.
Also, due to its generality, the tools depend heavily on the underlying system to generate traffic with specific characteristics.
Thus, the responsibility lies with the user to grasp both a practical and theoretical understanding of the tool and the underlying system.

\skippara In the previous and our work, we emphasize the importance to identify a specific purpose and choose the most suitable metrics and tools accordingly.
Our results show that the \acrshort{cpu} and \acrshort{nic} limit the throughput produced from the userspace tools for different packet sizes.
The results remain consistent alongside previous studies.
Besides, we run the userspace tools on top of a virtual switch and hardware and show that only the \acrshort{cpu} limits the tool's throughput performance.
Conclusively, the tools are useful for smaller end-to-end system tests.
Especially suitable, in combination with container technology to achieve higher reproducibility and automation capabilities in both research and industry.

\skippara In the future, it would be interesting to examine network tools that use high-speed packet processing libraries, for example, \acrfull{dpdk} on commodity hardware.
Also, to further investigate software tools that facilitate network virtualization, for example, Software-Defined Networking (SDN) technologies.
