\chapter{Introduction}\label{introduction}

\acrfull{ict} companies, for example, cloud service providers and mobile network operators provide reliable network products, services or solutions to handle network traffic on a large scale.
These companies rely on proprietary and hardware-based network testing tools to test their products before deployment.
That is, to generate realistic network traffic, which is then injected into a server to verify its behavior.
Because of high demand to send and receive information with high speed and low latency, it is essential to test the \gls{ict} solutions thoroughly.

%----------------------------------------------------------------
%
% Problem
%
%----------------------------------------------------------------
\section{Problem Statement}

In contrast to \gls{ict} enterprises, the network research community develops and uses open-source and software-based network testing tools.
Thus, researchers use software-based network testing tools \cite{DITGDist48:online, Packetge32:online, ToolsThe22:online} in experiments because of its flexibility and for economic reasons \cite{botta2010you, molnar2013validate}.
However, the generated network traffic from these tools is not as reliable as the one from the hardware-based platform, because of the underlying hardware and software, such as \gls{nic} and \gls{os}.
Use of the software-based tools without an awareness of these variables, can produce inaccurate results between the generated and requested network traffic.

\skippara This paper examines the software-based network traffic generators Iperf, Mausezahn, and Ostinato.
The purpose is to test the chosen tools concerning accuracy for different network profile, and the efficiency with lightweight hardware and software typical for academic environments.
Distinctively to other similar studies, this project uses the container technology Docker to encapsulate the tools for automated tests and to achieve a higher degree of a reproducibility \cite{piccolo2016tools, boettiger2015introduction, Chamberlain2014}.

%----------------------------------------------------------------
%
% Limitation
%
%----------------------------------------------------------------
\section{Limitation}
Our study only investigates software-based traffic generators that primarily operate in user space and uses the Linux networking stack.

%----------------------------------------------------------------
%
% Sustainability, Ethics, and Societal Aspects
%
%----------------------------------------------------------------
\section{Sustainability, Ethics, and Societal Aspects}
In modern life, the Internet is an integral part of the human environment, where stability, security, and efficiency are the essential aspects.
This project has no direct and significant impact on sustainability in general.
Except, the power consumption of laptops with the purpose to gather data.


\skippara From an ethical standpoint, we documented our steps throughout our thesis, provided the code in the appendices, and in a public repository \url{https://github.com/saimanwong/mastersthesis}.
That is, to contribute to reproducible research and transparency.
Hence, other people are encouraged to try to replicate and achieve similar results.
However, and most likely, the results can vary because of the underlying hardware and software.

\skippara Since this project only uses open source software, it is just morally right to make everything public.
Also, the network tools used in this project are purposely for private and controlled labs or virtual environments.
Thus, it is inadvisable to use these tools on public networks.
