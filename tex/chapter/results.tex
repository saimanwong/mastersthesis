\chapter{Results}\label{results}

We used the parameter settings in \cref{section:settings} to send \acrshort{udp} traffic with various packet sizes between two hosts in a physical and virtual environment.
In both environments presented in \cref{section:labenv}, the first host (source) generates and sends the traffic to the second host (sink) which captures it.
For each packet size varied between 64 and 4096 bytes, we ran 100 simulations and each run for 10 seconds.
The following couple of sections present the performance of the traffic generators concerning throughput.


\section{Physical Hardware}
\input{plot/experiment1}

\skippara As shown in \cref{fig:experiment_host}, Ostinato reached the highest throughput of 984.61$\pm$14.54 \acrshort{mbps} at packet size 4096 byte.
Mausezahn and Iperf reached their maximum throughput of 965.94$\pm$23.47 \acrshort{mbps} and 965.55$\pm$48.67 \acrshort{mbps} respectively, both at packet size 3072 byte.
Iperf had a slow start from 64 to 768 bytes in packet size.
That is, in contrast to Ostinato and Mausezahn that achieved similar throughput from 256 to 3072 bytes in packet size.
Additionally at packet sizes 3072 and 4096 bytes, Ostinato has a throughput sparseness (standard deviation) that is more than half compared to Mausezahn and Iperf.

\section{Virtual Hardware}
\begin{figure}[h!]
    \begin{adjustbox}{width=1\textwidth, center}
        \renewcommand*\arraystretch{1.5}
        \begin{tikzpicture}
            \begin{axis}[
                title={},
                legend style={font=\scriptsize},
                x label style = {text height = 0.7cm},
                xlabel={Packet Size in Bytes},
                ylabel={Throughput in Megabit Per Second},
                xmin=0, xmax=4096,
                ymin=0, ymax=2100,
                xtick={0, 1000, 2000, 3000, 4000, 5000},
                ytick={0, 200, 400, 600, 800, 1000, 1200, 1400, 1600, 1800, 2000, 2200},
                legend pos=south east,
                ymajorgrids=true,
                grid style=dashed,
                height=11cm,
                width=20cm,
                /pgf/number format/.cd,
                1000 sep={}
                ]

                \addplot [color=black, mark=o]
                plot [error bars/.cd, y dir = both, y explicit]
                table[y error index=2]{plot/vm_theoretical.dat};

                \addplot [color=blue, mark=o]
                plot [error bars/.cd, y dir = both, y explicit]
                table[y error index=2]{plot/vm_ostinato.dat};

                \addplot [color=red, mark=o]
                plot [error bars/.cd, y dir = both, y explicit]
                table[y error index=2]{plot/vm_mausezahn.dat};

                \addplot [color=green, mark=o]
                plot [error bars/.cd, y dir = both, y explicit]
                table[y error index=2]{plot/vm_iperf.dat};

                \legend{Theoretical, Ostinato, Mausezahn, Iperf}

            \end{axis}
        \end{tikzpicture}
    \end{adjustbox}
    \caption{Throughput Graph Summary in Virtual Environment}
    \label{fig:experiment_vm}
\end{figure}


\skippara \cref{fig:experiment_vm} illustrates the results from the virtual lab.
Mausezahn reached the highest throughput of 2023.85$\pm$23.47 \acrshort{mbps} at 4096 bytes in packet size.
Ostinato and Iperf achieved 1969.38$\pm$20.42 and 1957.25$\pm$48.67 \acrshort{mbps} maximum throughput at packet sizes 3072 and 4096 bytes respectively.
All the network tools keep relatively and similar throughput rate.
Except, Ostinato at packet size 3072 byte where it diverges from its maximum throughput.
