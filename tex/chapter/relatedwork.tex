\chapter{Related Work}\label{relatedwork}

In past research on traffic generators \cite{botta2010you, turull2016pktgen, kolahi2011performance, srivastava2014evaluation, srivastava2014comparative}, were conducted experiments in a closed-loop environment between two hosts connected via a link.
All these experiments included throughput test on traffic generators, in either packet per second or bit rate.
As mentioned in \cref{metrictype}, the network research community frequently uses the metrics byte throughput and packet size.
We try to focus on the following metrics in our project.
In \cite{kolahi2011performance, srivastava2014evaluation, srivastava2014comparative}, the authors conducted similar throughput experiments with userspace tools, and we used their method as inspiration for this project.

\skippara In \cite{kolahi2011performance}, the authors generated \acrshort{tcp} traffic over a 100 \acrshort{mbps} link, packet sizes ranging from 128 to 1408 bytes, and used the tools Iperf, Netperf, D-ITG and IP Traffic.
The authors in \cite{srivastava2014evaluation, srivastava2014comparative} generated both \acrshort{tcp} and \acrshort{udp} traffic over 10 and 40 \acrshort{gbps} links respectively, packet sizes between 64 and 8950 bytes, and with Iperf, PackETH, Ostinato, and D-ITG.
The results are similar in all these experiments, in the sense that the throughput increases for larger packet sizes because small packet sizes introduce higher overhead and cause lower performance.
\cref{table:experimentsummary} summarizes the highest throughput from these experiments.

\begin{table}[ht!]
    \scriptsize
    \caption{Maximal Throughput Summary of \cite{kolahi2011performance, srivastava2014evaluation, srivastava2014comparative}}
    \label{table:experimentsummary}
    \begin{adjustbox}{center}
        \renewcommand*\arraystretch{1.5}\begin{tabular}{| L{1.6cm} | L{1.7cm} | L{1.7cm} | L{1.7cm} || L{1.7cm} | L{1.7cm} |}

            \hline
            \textbf{Tools} & \textbf{\cite{kolahi2011performance} \acrshort{tcp}} & \textbf{\cite{srivastava2014evaluation} \acrshort{tcp}} & \textbf{\cite{srivastava2014comparative} \acrshort{tcp}} & \textbf{\cite{srivastava2014evaluation} \acrshort{udp}} & \textbf{\cite{srivastava2014comparative} \acrshort{udp}}
            \\ \hline
            D-ITG & 83.8 \acrshort{mbps} & 6520 \acrshort{mbps} & 16.2 \acrshort{gbps} & 7900 \acrshort{mbps} & 26.7 \acrshort{gbps}
            \\ \hline
            Iperf & \textbf{93.1 \acrshort{mbps}} & 5400 \acrshort{mbps} & - & 1.05 \acrshort{mbps} & -
            \\ \hline
            Ostinato & - & \textbf{9000 \acrshort{mbps}} & 38.8 \acrshort{gbps} & 9890 \acrshort{mbps} & 36.9 \acrshort{gbps}
            \\ \hline
            PackETH & - & 7810 \acrshort{mbps} & \textbf{39.8 \acrshort{gbps}} & \textbf{9980 \acrshort{mbps}} & \textbf{39.8 \acrshort{gbps}}
            \\ \hline
            IP Traffic & 76.7 \acrshort{mbps} & - & - & - & -
            \\ \hline
            Netperf & 89.9 \acrshort{mbps} & - & - & - & -
            \\ \hline \hline
            Iperf multithread & - & 9540 \acrshort{mbps} & - & 12.6 \acrshort{mbps} & -
            \\ \hline
            D-ITG multithread & - & 9820 \acrshort{mbps} & 39.9 \acrshort{gbps} & 8450 \acrshort{mbps} & 39.8 \acrshort{gbps}
            \\ \hline
        \end{tabular}
    \end{adjustbox}
\end{table}


\skippara In latterly mentioned experiments, we encountered two observations that could make it harder for us to reproduce.
First, they used network tools with either an architecture to both send and capture traffic or only send.
Therefore, it is unclear what method they used to monitor the traffic.
Second, we could not find which version of the network tools they used.
To avoid ambiguity and to make our project more reproducible, we packaged the tools into different Docker containers, which makes them easy to deploy on various of systems.
In our study, we try to evaluate the limits of userspace traffic generators used in Docker container environment.
Finally, \cite{felter2015updated, chung2016using} showed that Docker containers are feasible for data-intensive applications, and reach close to native \acrshort{cpu} and memory performance, because these share the same resources as the host \acrshort{os}.


\skippara Alternatively to our approach and as examined in \cref{metrictype}, other network testing tools operate in the kernel space, or use external libraries to circumvent the host kernel and directly access commodity \acrshort{nic} to achieve faster packet processing.
In \cite{turull2016pktgen}, the authors developed the kernel module and traffic generator called Pktgen.
They ran throughput, latency and packet delay variation experiments between two machines to compare their tool to the userspace tools Iperf and Netperf; as well as Pktgen-\acrshort{dpdk} and Netmap, which use an external framework to process packets, such as \acrshort{dpdk}.
In the maximal throughput experiment for \acrshort{udp} traffic at 64 bytes packet size, former tools achieved between 150 and 350 \acrshort{mbps}, Pktgen generated a throughput of 4200 \acrshort{mbps}, and both the latter types saturated the link at 7600 \acrshort{mbps}.
MoonGen is another high-speed and \acrshort{dpdk}-based traffic generator.
The creators of MoonGen generated \acrshort{udp} traffic at minimal packet size and also saturated a 10 \acrshort{gbps} link at a lower clock frequency than Pktgen-\acrshort{dpdk} \cite{emmerich2015moongen}.
We compare and summarize our method to other related studies in \cref{tab:comparesummary}.


\skippara All the related studies used the similar experimental methodology to examine software generators.
That is, to conduct tests in an isolated environment of two machines to avoid external influences.
Naturally, traffic generators built closer to the hardware perform significantly better than userspace tools.
However, the hardware and software become more adaptable and cheaper; we decided to further examine the performance of userspace tools and their use cases.
Also, to enable \acrshort{devops}\footnote{\gls{devops} -- It is a software development philosophy, which a large number of automation and monitoring tools arose.
Of which, Docker is one of the leading tools on the market.
The author in \cite{boettiger2015introduction} describes the \acrshort{devops} methodology as: ``The approach is characterized by scripting, rather than documenting, a description of
the necessary dependencies for software to run, usually
from the Operating System (\acrshort{os}) on up.''
} practices with container technology to more efficiently automate tests in a standardized manner.

\clearpage

\begin{table}
\caption{Comparison and Summary Table of Related Work}
\label{tab:comparesummary}
\centering
{\tiny\begin{tabular}{|>{\centering\arraybackslash}m{9mm}|m{30mm}|m{35mm}|m{55mm}|}

\hline
\multicolumn{1}{|>{\centering\arraybackslash}m{9mm}|}{}
& \multicolumn{1}{>{\centering\arraybackslash}m{30mm}|}{\textbf{Traffic Generator}}
& \multicolumn{1}{>{\centering\arraybackslash}m{35mm}|}{\shortstack[c]{\\ \textbf{Metrics} \\ frequently used \\ in literature indentified \\ by \cite{molnar2013validate}}}
& \multicolumn{1}{>{\centering\arraybackslash}m{55mm}|}{\textbf{Lab Environment}}\\
        \hline
        2017 Our Work
        & \shortstack[l]{\\ User Space \\ $-$ Iperf \\ $-$ Mausezahn \\ $-$ Ostinato}
        & \shortstack[l]{\\ $-$ Byte Throughput \\ $-$ Packet Size \\ \hspace{2.5mm} Distribution}
        & \shortstack[l]{\\ Physical Lab \\ $-$  Intel Core i5-2540M \acrshort{cpu} \\ $-$ 8 GB Memory \\ $-$ 1000 \acrshort{mbps} \acrshort{nic} \\ $-$ Ubuntu Server 16.04.2 \\ $-$ Linux kernel version 4.4.0 \\ $-$ Docker version 17.05.0-ce \\ \ \\ Virtual Lab \\ \quad Host \\ \qquad $-$ Intel Core i5-5257U \acrshort{cpu} \\ \qquad $-$ 8 GB Memory \\ \qquad $-$ macOS 10.12.5 \\ \qquad $-$ Oracle VM VirtualBox 5.1.22 \\ \quad Guest \\ \qquad $-$ 1 CPU \\ \qquad $-$ 2 GB Memory \\ \qquad $-$ Ubuntu server 16.04.2 \\ \qquad $-$ Linux kernel version 4.4.0 \\ \qquad $-$ Docker version 17.05.0-ce}
        \\
        \hline
        \hline
        2016 \cite{turull2016pktgen}
        & \shortstack[l]{\\ User Space \\ $-$ Netperf \\ $-$ Iperf \\ \ \\ Kernel Space \\ $-$ Pktgen \\ \ \\ External Framework \\ $-$ Netmap  \\ $-$ Pktgen-\acrshort{dpdk}}
        & \shortstack[l]{\\ $-$ Byte Throughput \\ $-$ Packet Size \\ \hspace{2.5mm} Distribution \\ $-$ Inter Packet Time \\ \hspace{2.5mm} Distribution}
        & \shortstack[l]{\\ Physical Lab \\ $-$ Intel Xeon \acrshort{cpu} \\ $-$ 3 GB Memory \\ $-$ 10 \acrshort{gbps} \acrshort{nic} \\ $-$ Ubuntu 13.10 \\ $-$ Linux kernel version 3.18.0}
        \\
        \hline
        \hline
        2015 \cite{emmerich2015moongen}
        & \shortstack[l]{\\ External Framework \\ $-$ MoonGen  \\ $-$ Pktgen-\acrshort{dpdk}}
        & \shortstack[l]{\\ $-$ Byte Throughput \\ $-$ Packet Size \\ \hspace{2.5mm} Distribution \\ $-$ Inter Packet Time \\ \hspace{2.5mm} Distribution}
        & \shortstack[l]{\\ Physical/Virtual Lab \\ $-$ Intel Xeon \acrshort{cpu} \\ $-$ 1/10/40 \acrshort{gbps} \acrshort{nic} \\ $-$ Debian \\ $-$ Linux kernel version 3.7 \\ $-$ Open vSwitch 2.0.0}
        \\
        \hline
        \hline
        2014 \cite{srivastava2014comparative}
        & \shortstack[l]{\\ User Space \\ $-$ D-ITG  \\ $-$ Ostinato \\ $-$ PackETH}
        & \shortstack[l]{\\ $-$ Byte Throughput \\ $-$ Packet Size \\ \hspace{2.5mm} Distribution}
        & \shortstack[l]{\\ Physical Lab \\ $-$ Intel Xeon \acrshort{cpu} \\ $-$ 40 \acrshort{gbps} \acrshort{nic} \\ $-$ CentOS 6.5 \\ $-$ Linux kernel version 2.6.32}
        \\
        \hline
        \hline
        2014 \cite{srivastava2014evaluation}
        & \shortstack[l]{\\ User Space \\ $-$ D-ITG  \\ $-$ Iperf \\ $-$ Ostinato \\ $-$ PackETH}
        & \shortstack[l]{\\ $-$ Byte Throughput \\ $-$ Packet Size \\ \hspace{2.5mm} Distribution}
        & \shortstack[l]{\\ Physical Lab \\ $-$ Intel Xeon \acrshort{cpu} \\ $-$ 64 GB Memory \\ $-$ 10 \acrshort{gbps} \acrshort{nic} \\ $-$ CentOS 6.2 \\ $-$ Linux kernel version 2.6.32}
        \\
        \hline
        \hline
        2011 \cite{kolahi2011performance}
        & \shortstack[l]{\\ User Space \\ $-$ D-ITG  \\ $-$ IP-Traffic \\ $-$ Iperf \\ $-$ Netperf}
        & \shortstack[l]{\\ $-$ Byte Throughput \\ $-$ Packet Size \\ \hspace{2.5mm} Distribution}
        & \shortstack[l]{\\ Physical Lab \\ $-$ Intel Pentium 4 \acrshort{cpu} \\ $-$ 1 GB Memory \\ $-$ 100 \acrshort{mbps} \acrshort{nic} \\ $-$ Windows Server 2003}
        \\
        \hline
        \hline
        2010 \cite{botta2010you}
        & \shortstack[l]{\\ User Space \\ $-$ D-ITG  \\ $-$ MGEN \\ $-$ RUDE/CRUDE \\ $-$ TG}
        & \shortstack[l]{\\ $-$ Byte Throughput \\ $-$ Packet Size \\ \hspace{2.5mm} Distribution \\ $-$ Inter Packet Time \\ \hspace{2.5mm} Distribution}
        & \shortstack[l]{\\ Physical Lab \\ $-$ Intel Pentium 4 \acrshort{cpu} \\ $-$ 1 \acrshort{gbps} \acrshort{nic} \\ $-$ Linux kernel version 2.6.15}
        \\
        \hline
\end{tabular}}
\end{table}

