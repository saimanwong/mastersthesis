\chapter{Discussion}\label{discussion}

The network research community uses software-based traffic generators for a variety of purposes because these are often open sourced and flexible to handle.
In this project, we examined open source traffic generators that primarily operate in the user space and use the Linux default network stack.
The authors in \cite{kolahi2011performance, srivastava2014evaluation, srivastava2014comparative} examined these type of generators, concerning throughput with varied packet sizes, over 100 \acrshort{mbps}, 10 \acrshort{gbps}, and 40 \acrshort{gbps} links.
We presented a summary of these past research experiments in \cref{relatedwork} and \cref{table:experimentsummary}.
We conducted similar benchmark experiments and achieved traffic characteristics comparable to theirs, but over a 1000 \acrshort{mbps} link.
We also used container technology, Docker, to package this experiment and make it simpler to replicate.

\section{Performance Evaluation}

As expected from previous studies, our results in \cref{results} (\cref{fig:experiment_host} and \cref{fig:experiment_vm}) show that the throughput increased with the packet size.
All the traffic generators yield lower throughput on smaller packet sizes because the \acrshort{cpu} is put on high workload when it has to generate these smaller packets.
It is an expensive operation to copy data between user and kernel space on top of the Linux default network stack \cite{raumer2015performance}.
Mainly, the experimental results are approximately between $12\%$ and $75\%$ below the theoretical throughput for 64 to 512 bytes in packet size, as shown in \cref{fig:experiment_host}.
Thus, the \acrshort{cpu} reaches its full capacity before it saturates the link.

\skippara In both labs, we noticed in broad strokes that the standard deviation also increases with the packet size, but there are exceptions.
For example, Ostinato generated the highest throughput and lowest standard deviation in the physical lab.
It yielded 985.61$\pm$14.54 \acrshort{mbps} at 4096 bytes packet size.
That is, in contrast to Mausezahn and Iperf that generated 965.94$\pm$34.83 and 965.55$\pm$33.77 \acrshort{mbps}, respectively, both at 3072 bytes packet size.
We send multiple packets at once with Ostinato in burst mode instead of single.
It might explain why Ostinato generated the highest throughput more consistently than Mausezahn and Iperf.

\skippara Our results in the physical lab reach consistent and similar traffic characteristics as the previous research, \cite{kolahi2011performance, srivastava2014evaluation, srivastava2014comparative} individually.
That is, the traffic generators directly use the host's \acrshort{nic} to saturate the link for larger packet size, and \acrshort{cpu} limits throughput for smaller packet sizes.
However, the results in the lab environment with virtual hardware differed from the physical one, as the tools generated over 1900 \acrshort{mbps} throughput.
The primary reason is that we use ``internal networking''-mode \cite{Chapter680:online}.
That is, packets flow via a virtual network switch instead of the host and its \acrshort{nic}.
Thus, the packet processing and maximum throughput depended entirely on the host's \acrshort{cpu} thereof, the higher performance.

\skippara We also encountered another unanticipated result when we experimented with Iperf.
In \cite{srivastava2014comparative}, the authors examined the throughput of Iperf with multiple threads.
Their \acrshort{udp} traffic did not saturate the 10 \acrshort{gbps} link ranged.
That is, it generated from 1.05 \acrshort{mbps} to 12.6 \acrshort{mbps}, with one respectively twelve threads.
However, our results indicate that Iperf with one thread keeps up with the throughput of both Ostinato and Mausezahn.
We investigated it and found that Iperf 2.0.5 produced unexpected throughput on various packet sizes, similar to the latter mentioned authors' findings.
Thus, we used Iperf 2.0.9, a newer version, with one thread in this project.

\skippara Altogether, we wanted to examine the traffic generators in two different environments.
One might argue that the lab with physical hardware yielded more realistic results compared to the virtual environment because we only used the real equipment.
On the other side, sometimes resources are limited, then virtual environments on a single host might be a more feasible option to experiment within.
However, in the end, it comes down to picking the most suitable tool for the job.
It sounds simple, but it is more difficult to apply in practice, because there is a large number of traffic generators with different purposes, and we will further discuss this in \cref{discuss:tgme}.

\section{Experiment Evaluation}

We used the closed-loop approach to experiment with the traffic generators.
It is a standard approach to test the behavior of both traffic generator and underlying hardware.
That is, it consists of two directly connected hosts in a controlled and isolated environment to minimize external influences.
Besides, we packaged the software into standardized images/Dockerfile(s), which facilitated the work for us when we were to, for example, tune the parameters and move between different environments.
This approach had an insignificant impact on performance as the containers had direct and privileged access to the kernels resources.

\subsection{Reproducible Research}
As for reproducible research, there is a challenge to replicate the exact results, because these network tools heavily depend on the underlying system.
However, we use container technology to achieve a reproducible research partly.
That is, we wrote Dockerfile(s) that the user can build an image from, or make modifications to fit a specific purpose.
Furthermore, we provide a bash-script to automatically spin up containers of these tools on two remote servers, either physical or virtual one.
The only requirement is that the servers have Docker installed on these servers.

\subsection{Validation -- Traffic Generators and Metrics}\label{discuss:tgme}
Our literature review in \cref{sec:tg} delves into why researchers use software-based traffic generators, the different types, metrics, and challenges with them.
However, in this section, we try to discuss our procedure, to raise awareness and maybe help other people to choose a suitable network tool or several tools.
Additionally, we specifically discuss around and delve into \cite{botta2010you, molnar2013validate}, where the authors critically assessed a large number of traffic generators from different sources.

\skippara First off, we expect a traffic generator to send traffic with specific characteristics.
We can interpret these characteristics as different kind of requirements.
When we talk about accuracy, it is about how well the generated synthetic traffic matches with the specified requirements and error margins.
Thus, it is equally as important to consider which metric(s) to experiment around, as it is to pick the network tool that can fulfill these.

\skippara Since there are a large number of software-based traffic generators, the authors in \cite{botta2010you, molnar2013validate} questioned the methods to validate the accuracy and reliability of these tools in the network community,
notably, the traffic generators that primarily operate in user space on top of any arbitrary \acrshort{os} and network stack.
That is, given the tool's generality and flexibility, it requires from the researcher(s) a greater understanding of the underlying system to produce accurate results, which is often a challenge that gets minimal attention in the literature.

\skippara In regards to our findings, we used the metrics byte throughput and packet size, a similar method to \cite{kolahi2011performance, srivastava2014evaluation, srivastava2014comparative} to try to experiment with two recently active network tools (Ostinato and Iperf) and an inactive one (Mausezahn).
We showed that these tools saturate the link at packet sizes above 4096 bytes.
If the only purpose is to test end-to-end system performance regarding maximum throughput, then these tools can be a reasonable choice.
However, our and \cite{kolahi2011performance, srivastava2014evaluation, srivastava2014comparative} studies do not examine other parameters, such as packet delay, loss, and fragmentation.
As in the previously mentioned papers, we want to suggest to carefully consider which network tool is the most suitable to user's requirements.

\clearpage
\subsection{Recommendations}

The tools we evaluated are flexible to use in a broad variety of environments but can yield different results depending on the circumstance.
Thus, we recommend applying these tools in a smaller scale and low-risk environments as complements for hardware-based traffic generators.
For example, for personal experiments and small networks with more straightforward traffic characteristics.

\skippara We recommend Ostinato and Mausezahn for users that want to test and create any arbitrary packets from layer 2 and up.
In this case, the former is a more flexible choice as it is available cross-platform, has a \acrshort{gui} and Python-\acrshort{api} with automation capabilities.
For general bandwidth test, we recommend Iperf because the creators developed it for this specific purpose.
Finally, we recommend a more profound research into network tools that use libraries specialized for fast packet processing on commodity hardware.
